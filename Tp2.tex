\documentclass{article}
\usepackage[utf8]{inputenc}

\title{IFT2035 - TP 2}
\author{David Georgaris (20160503) \and Adel Abdeladim (20127626) }
\date{12 Décembre 2022}


\usepackage{natbib}
\usepackage{graphicx}
\usepackage{amsmath}
\usepackage{dsfont}
\usepackage{adjustbox}
\usepackage{tabularx}
\usepackage{seqsplit}

\usepackage[left=20mm, right=20mm, top=20mm, bottom=20mm]{geometry}

\begin{document}

\maketitle
\section{Code de base}
    Nous avons commencé par lire et comprendre les différentes étapes 
    qu'emprunterait le code, de sa lecture à sa sortie en valeur. Les 
    deux étapes principales sont la compilation de l'expression régulière
    en NFA suivie du parcours du NFA afin de matcher la chaîne de caractères 
    donnée en entrée avec l'expression régulière. \\


\section{nfa\_match}
    \subsection{À faire}
        La fonction $nfa\_match$ va prendre un NFA et va suivre les valeurs d'entrées de Str, 
        les Steps, jusqu'à atteindre un état terminal, les chars restants de la Str seront retournés 
        dans Tail, Marks selon les marques qui ont été passés et Gs (Groups), les sous-groupes associés.  

    \subsection{Exécution}
        
        Dans le cas d'un match, on fait l'appel à $nfa\_search\_step$ qui va chercher le 
        $Step$ associé à l'état dans le NFA. Une fois ce $Step$ retourné, on fait un appel récursif à 
        $nfa\_match$ avec le nouveau $Step$ en entrée. \\
        Une fois la fin de l'appel consécutif récursif, on fait un appel à $nfa\_combine\_groups$, cette 
        fonction auxiliaire se charge de rajouter le $Char$ aux groupes obtenus récursivement. Si 
        cette fonction est appelée à partir d'un état Epsilon, on ajoute la chaine vide à tous les groupes
        afin de s'assurer que l'on puisse print une chaîne vide lorsque $nfa\_match$ se termine sans avoir 
        accepté un seul caractère. Dans le cas où la chaine n'est pas vide, on l'ajoute à la chaîne contenue dans $GroupsRec$. 
        Après l'avoir ajoutée, on la supprime ensuite de $GroupsRec$ pour éviter de doubler la chaine, lors de l'ajout final 
        au retour des appels récursifs, puis on retourne $GOut$ comme étant la chaine des groupes finaux.\\
        Dans le cas où le $char$ requis par l'état pour changer ne match pas avec la lecture faite 
        du $char$ dans le $String$, on passe simplement au prochain caractère. \\
        On va aussi prendre en compte les situations où on lit un état en fin de liste ou un état comme $Step$ 
        seul; dans ces deux cas, on fait les étapes de bases simplement, à savoir que l'on va chercher le $step$ 
        avant de rappeler la récursion.\\
        Les cas les moins instinctifs, à savoir les cas de lecture d'$Epsilon$, sont gérés dans l'appel de $nfa\_marks$, 
        avant de passer aux appels de base, à savoir le $Search$ suivi de la récursion\\
        La fonction $nfa\_marks$ se charge donc de prendre les $OldMarks$ déjà vues et modifie la liste en fonction des 
        $NewMarks$ rencontrés et si le $Mark$ est un $end(C)$, on retire le $beg(C)$ de la liste et le $Mark$ C n'est 
        donc plus actif.\\
        Les problèmes rencontrés au cours de la rédaction de cette fonction sont surtout reliés aux deux différentes façons
        d'effectuer de la récursion. Le premier, celui de la gestion de $Tail$, ressemblait plus à un appel séquentiel : 
        on devait simplement appeler la fonction en lui donnant le reste de $Str$ et le cas de base contient le résultat renvoyé.
        Inversement, pour gérer les $Groups$, le cas de base dictait que la liste devait être vide 
        lors du dernier appel de $nfa\_match$. On devait ainsi écrire une nouvelle fonction qui calculait récursivement les $Groups$
        de façon à ce qu'on construise le résultat de la fonction à partir de la liste vide du cas récursif.\\
        Un problème amusant qui nous est aussi arrivé est un appel récursif sans fin qui causait un $Stack Overflow$. 
        La solution était simplement d'inverser deux appels de fonctions afin de s'assurer que celle qui bouclait indéfiniment
        ait tous ses arguments et puisse arriver au cas de base.


\section{re\_comp}
\subsection{À faire}

   Cette fonction sert à compiler une expression régulière sous la forme d'un
   NFA. Ce qui nous était donné couvrait : le cas de base, la concaténation de 
   deux expressions régulières ainsi que les règles $'*'$ et $'.'$ . Pour 
   compléter cette fonction, il faut donc implémenter les règles des expressions 
   régulières restantes ainsi que le cas où notre expression est un simple caractère.
    
 \subsection{Exécution}
 
    Cette fonction était plutôt simple à compléter puisque les cas déjà faits 
    dans le code source offraient une bonne compréhension du fonctionnement de
    la fonction. Par exemple, le cas de concaténation d'expressions régulières 
    montrait bien comment enchaîner plusieurs appels récursifs en utilisant 
    le noeud intermédiaire. Les cas des règles $'.'$ et $'*'$ montraient bien quand 
    initialiser un état et comment effectuer un appel récursif dans un ordre 
    autre que de $B$ à $E$.\\
    Le cas $'+'$ était plutôt simple à gérer comme on pouvait se baser sur le 
    cas équivalent de la fonction $match$. Il fallait simplement forcer 
    l'application de la règle RE une fois de $B$ vers $I$, puis appeler 
    récursivement $*RE$ de $I$ vers $E$. De cette façon, on s'assure que la règle 
    soit appliquée au moins une fois. \\
    Le cas du $'?'$ était aussi très similaire au cas $'*'$ déjà donné. Il 
    suffisait de déclarer un état intermédiaire $B1$ et permettre de se 
    déplacer de B à E en passant soit par $B1$ avec l'expression régulière, 
    soit directement avec une transition epsilon. \\
    Pour le cas du $'\/'$, on pouvait forement s'inspirer de la concaténation 
    en introduisant deux états intermédiaires et une transition epsilon de 
    $B$ vers ces deux états. Ainsi, on s'assure que l'une des deux expressions 
    régulières sera respectée. \\
    Les cas $'in'$ et $'notin'$ étaient légèrement plus complexes. Pour y arriver, 
    nous avons dû créer une nouvelle fonction qui prend la chaîne de 
    caractères contenue et construit une liste de $steps$ qui se rendent de $B$ 
    vers l'état donné en utilisant chacun des caractères. Le $'notin'$ était 
    légèrement plus complexe comme on devait plutôt envoyer vers un état 
    terminal autre que $E$. Au départ, comme nous avions mal compris que la liste 
    contenue dans $steps()$ pouvait se terminer par un noeud, ce qui signifiait 
    que si le caractère n'était pas dans la liste, alors l'étape suivante serait 
    ce dernier état. Pour résoudre ce problème, nous avions écrit une fonction 
    auxiliaire peu chic qui prenait l'ensemble des caractères ASKII et qui 
    enlevait les caractères contenus dans le $'notin'$. Cette liste était ensuite 
    donnée à un cas $'in'$. Lorsque le professeur a mentionné que l'on pouvait 
    simplement mettre un noeud à la fin de la liste de $steps()$, cela a donc 
    beaucoup simplifié notre NFA.\\
    Finalement, le cas du caractère seul était très simple après avoir 
    construit tous les autres cas. Il fallait ajouter un $step()$ contenant 
    une transition de $B$ vers $E$ avec ce caractère.
    
 
\section{nfa\_epsilons}
\subsection{À faire}

    La fonction $nfa\_epsilons$ prend un NFA et a pour but d'éliminer les chaînes 
    de transitions epsilon pour éviter d'avoir des cycles infinis. La fonction 
    était majoritairement complétée dans le code source. 
    Il fallait simplement la modifier pour qu'elle accumule toutes les Marks 
    visitées lors de son parcour et mette cette liste dans l'état $S$ initial.

\subsection{Exécution}
    Au départ, comme la fonction n'était même pas mentionnée dans la donnée, 
    nous pensions qu'il n'était pas nécessaire de la compléter. Cependant, 
    lorsque nous tentions de compiler une expression régulière contenant un 
    $name(k, RE)$ , les marques n'apparaissaient pas dans le NFA retourné. 
    Il était donc évident que nous devions compléter cette fonction. \\
    Les premières tentatives étaient toutes des échecs. Nous avions essayé de 
    créer une fonction auxiliaire qui extrayait les $Marks$ visitées, modifier 
    la fonction $nfa_epsilons\_1$ pour qu'elle accumule aussi la liste de $Marks$ 
    visitées ou même de dupliquer la fonction $nfa_epsilons\_1$ et de modifier 
    cette nouvelle version. Après beaucoup trop de grattage de tête, nous 
    avions abandonné et décidé que seuls les noeuds ne contenant pas de 
    $Marks$ seraient élimminés. Quelques jours plus tard, nous avons réessayé 
    de corriger cette fonction et, pour une raison que nous ignorons, la 
    correction a fonctionné du premier coup. 

\end {document}
